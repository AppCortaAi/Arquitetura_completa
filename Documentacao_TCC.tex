\documentclass[12pt, a4paper]{report}

% Configurações Essenciais
\usepackage[utf8]{inputenc} % Codificação para acentos e cedilha
\usepackage[T1]{fontenc}
\usepackage[brazil]{babel} % Regras de hifenização e datas em português
\usepackage{amsmath, amsfonts, amssymb} % Pacotes matemáticos
\usepackage{graphicx} % Para incluir imagens (diagramas, gráficos)
\usepackage{longtable} % Para tabelas que podem quebrar páginas
\usepackage{booktabs} % Para linhas profissionais em tabelas (\toprule, \midrule, \bottomrule)
\usepackage{url} % Para formatar links (URLs) nas referências

% Configuração de Margens (Exemplo ABNT: Topo 3cm, Esquerda 3cm, Direita 2cm, Baixo 2cm)
\usepackage[
  a4paper,
  top=3cm,
  bottom=2cm,
  left=3cm,
  right=2cm,
  headheight=15pt,
  bindingoffset=0mm
]{geometry}

\begin{document}

\title{\textbf{Documento de Projeto}}
\author{Equipe CortaAi}
\date{\today}

\maketitle
\clearpage
\tableofcontents
\clearpage

% ##############################################################################
% 1 RESUMO
% ##############################################################################
\chapter{RESUMO}
\label{ch:resumo}

Este trabalho traz a proposta do desenvolvimento de um aplicativo web que facilite o funcionamento de uma ou mais barbearias, funcionando como um marketplace.
Muitas das pessoas perdem horas em filas para cortar seu cabelo ou fazer a barba, o que demonstra que o fluxo pode e deve ser melhorado, tanto visando a maior satisfação do cliente quanto um melhor gerenciamento da barbearia como um negócio, além de trazermos a plataforma como forma de otimizar a procura de profissionais qualificados, horários disponíveis e serviços prestados, temos também como objetivo realizar relatórios que transformem a administração do negócio algo intuitivo e fácil de fazer.

Visando a conclusão do projeto, adotamos o uso das metodologias que garantam a estruturação e eficácia do gerenciamento e implementação das funcionalidades essenciais da aplicação


Este trabalho traz a proposta do desenvolvimento de um aplicativo web de gestão de serviços e agendamentos que funcione como um marketplace para uma ou mais barbearias. O objetivo principal é eliminar ineficiências e desorganização, transformando a gestão do fluxo de clientes.
Historicamente, muitos clientes perdem horas em longas filas para cortar o cabelo ou fazer a barba, o que demonstra que o fluxo de atendimento pode e deve ser melhorado. Essa ineficiência prejudica tanto a satisfação do cliente quanto o gerenciamento da barbearia como um negócio.
O projeto surge como uma plataforma robusta para otimizar essa experiência. Além de ser um portal centralizado que facilita a procura por profissionais qualificados, horários disponíveis e serviços prestados em diferentes unidades, temos como objetivo primário automatizar o processo de agendamento.
A aplicação irá gerar relatórios intuitivos e detalhados sobre desempenho, finanças e estoque, transformando a administração do negócio em algo fácil e baseado em dados. Espera-se que essa solução resulte em uma melhoria significativa na experiência do usuário (cliente e barbeiro) e na eficiência operacional das barbearias parceiras.
