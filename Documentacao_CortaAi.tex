%%%%%%%%%%%%%%%%%%%%%%%%%%%%%%%%%%%%%%%%%%%%%%%%%%%%%%%%%%
%                                                        %
%  ARQUIVO LATEX PARA O TCC "CortaAi"                    %
%  Baseado no padrão abntex2                             %
%                                                        %
%%%%%%%%%%%%%%%%%%%%%%%%%%%%%%%%%%%%%%%%%%%%%%%%%%%%%%%%%%

% --- PREÂMBULO ---
\documentclass[
    12pt, 
    a4paper, 
    oneside, 
    chapter=TITLE, 
    section=TITLE
]{abntex2}

% --- PACOTES BÁSICOS ---
\hypersetup{
    colorlinks=true,
    linkcolor=black,
    urlcolor=blue,
    citecolor=black}
\usepackage[utf8]{inputenc}
\usepackage[brazil]{babel}
\usepackage{graphicx}       % Para inclusão de imagens
\usepackage{booktabs}       % Para tabelas com linhas profissionais
\usepackage{longtable}      % Para tabelas que quebram páginas
\usepackage{amsmath}        % Para símbolos matemáticos (como no DER)
% --- INFORMAÇÕES DA CAPA E FOLHA DE ROSTO ---
\titulo{CortaAi}
\autor{
    ANA CLARA MARTINS VIEIRA (SP3120104) \and
    GUILHERME BARBOSA CHAVES DA SILVA \and
    JOÃO PAULO MOTTA SAMPAIO (SP3128172) \and
    LARYSSA GIOVANNA MENDONÇA DE OLIVEIRA (SP3127435) \and
    KAIQUE ZEFERINO DINIZ (SP3126811) \and
    RENAN LADISLAU DE SOUSA (SP312651X)
}
\instituicao{
    IFSP - Instituto Federal de Educação, Ciência e Tecnologia
    \par
    Câmpus São Paulo
    \par
    Tecnologia em Análise e Desenvolvimento de Sistemas
}
\data{São Paulo - SP, 2025}
\tipotrabalho{Trabalho de Conclusão de Curso (TCC)}
\orientador{Marcelo Tavares de Santana}

% --- INÍCIO DO DOCUMENTO ---
\begin{document}

% --- ELEMENTOS PRÉ-TEXTUAIS ---
\imprimircapa
\imprimirfolhaderosto

% --- RESUMO ---
\begin{resumo}
Este trabalho traz a proposta do desenvolvimento de um aplicativo web de gestão de serviços e agendamentos que funcione como um marketplace para uma ou mais barbearias. O objetivo principal é eliminar ineficiências e desorganização, transformando a gestão do fluxo de clientes. Historicamente, muitos clientes perdem horas em longas filas para cortar o cabelo ou fazer a barba, o que demonstra que o fluxo de atendimento pode e deve ser melhorado. Essa ineficiência prejudica tanto a satisfação do cliente quanto o gerenciamento da barbearia como um negócio. O projeto surge como uma plataforma robusta para otimizar essa experiência. Além de ser um portal centralizado que facilita a procura por profissionais qualificados, horários disponíveis e serviços prestados em diferentes unidades, temos como objetivo primário automatizar o processo de agendamento. A aplicação irá gerar relatórios intuitivos e detalhados sobre desempenho, finanças e estoque, transformando a administração do negócio em algo fácil e baseado em dados. Espera-se que essa solução resulte em uma melhoria significativa na experiência do usuário (cliente e barbeiro) e na eficiência operacional das barbearias parceiras.

\vspace{\onelineskip}
\noindent \textbf{Palavras-chave:} Gestão de Barbearia. Agendamento Online. SaaS. Marketplace.
\end{resumo}

% --- SUMÁRIO ---
\tableofcontents
\newpage

% --- ELEMENTOS TEXTUAIS ---
\textual

% --- CAPÍTULO 1: INTRODUÇÃO ---
\chapter{Introdução}
\section{Objetivos}
O setor de serviços de cuidados pessoais, embora tradicionalmente fundamentado na interação presencial, atravessa um momento de transformação digital acelerada. Os consumidores modernos exigem conveniência, personalização e, acima de tudo, previsibilidade. Modelos de negócio que dependem exclusivamente de atendimento por ordem de chegada (walk-in) e gestão manual estão se tornando operacionais e financeiramente insustentáveis.

O ambiente clássico das barbearias, apesar de seu valor cultural, exemplifica essa inefiência crônica: é uma experiência comum clientes desperdiçarem horas em filas de espera, sem garantia de atendimento ou visibilidade sobre a disponibilidade de seus profissionais preferidos. Este cenário de espera e imprevisibilidade gera atrito significativo na jornada do cliente, resultando em insatisfação e perda de fidelidade. Para o gestor da barbearia, essa abordagem manual se traduz em um fluxo de trabalho caótico, dificuldade no gerenciamento das agendas dos profissionais, perda de receita de clientes que desistem da espera e uma incapacidade crônica de tomar decisões estratégicas por falta de dados consolidados. A administração do negócio torna-se reativa, focada em "apagar incêndios" operacionais, em vez de focar no crescimento.

Diante desse desafio, este trabalho propõe o desenvolvimento do CortaAi: uma aplicação web integrada, projetada para funcionar como um marketplace robusto para a indústria de barbearia. A solução transcende a funcionalidade de um simples aplicativo de agendamento, criando um ecossistema digital que conecta múltiplas unidades de barbearias, profissionais qualificados e clientes em uma plataforma centralizada e eficiente.

O objetivo primário do projeto é reestruturar o fluxo de atendimento, substituindo a fila física pela previsibilidade do agendamento digital. Através da plataforma, o cliente poderá pesquisar barbearias por localização, consultar o portfólio de serviços oferecidos, verificar as especialidades e a agenda de cada barbeiro, e reservar um horário (o agendamento) de forma instantânea. Para os barbeiros, o sistema oferece uma agenda digital clara, otimizando seu tempo e permitindo que se concentrem na qualidade do serviço.

Contudo, a principal inovação do CortaAi reside em sua capacidade de gestão de negócios (Business Intelligence). Além de organizar os agendamentos, o sistema foi modelado para integrar o controle de vendas de produtos e a gestão de estoque por unidade. Ao centralizar todas as transações, sejam elas de serviços prestados ou de produtos vendidos, a plataforma gerará relatórios intuitivos sobre o desempenho financeiro, a produtividade dos profissionais, os horários de pico e a receita por serviço. Isso transforma a administração do negócio de algo baseado na intuição para uma gestão baseada em dados, permitindo que os proprietários otimizem seus recursos, melhorem a lucratividade e profissionalizem suas operações.

\section{Problema e Solução Proposta}
A gestão operacional de barbearias, mesmo as que aparentam simplicidade, envolve um ecossistema complexo de processos que vão além do serviço de corte ou barba. O obstáculo central para a manutenção do negócio é a fragmentação das ferramentas de gestão. Atualmente, o processo administrativo demanda um esforço manual elevado e recorre a múltiplas ferramentas desconectadas:
\begin{itemize}
    \item Agendas Físicas (Cadernos): Para o controle de horários, suscetíveis a rasuras, agendamentos duplicados e dificuldade de consulta remota pelos barbeiros.
    \item Planilhas (Excel/Google Sheets): Para o controle financeiro (fluxo de caixa) e, por vezes, para o controle de insumos, sem integração automática com as vendas.
    \item Notas Fiscais (Blocos de Papel ou Emissores Avulsos): Sem ligação direta com os serviços prestados ou produtos vendidos.
    \item Aplicativos de Mensagens (WhatsApp): Usados de forma ineficiente para o agendamento, gerando interrupções constantes e perda de informações.
\end{itemize}
Esse sistema fragmentado gera desafios operacionais críticos, como perda de informações financeiras, dificuldade na gestão de insumos (estoque), agendamentos conflitantes e ausência de métricas de desempenho.

A solução proposta é o desenvolvimento do CortaAi, uma plataforma tecnológica centralizada que visa substituir e unificar todas as tarefas de gestão fragmentadas. O núcleo do projeto será uma plataforma SaaS (Software as a Service) de alta disponibilidade e usabilidade, operando em um ambiente único e integrado. O sistema funcionará como o cérebro operacional da barbearia, fazendo a interface direta com:
\begin{itemize}
    \item Módulo de Agendamentos: Permitindo a reserva online pelo cliente e a visualização da agenda em tempo real pelo barbeiro.
    \item Módulo Financeiro (PDV): Registrando cada serviço prestado e cada produto vendido.
    \item Módulo de Compras (Estoque): Controlando os insumos utilizados nos serviços e o inventário de produtos para revenda.
    \item Módulo de Pagamento: Integrando-se a gateways de pagamento para automatizar o recebimento e a conciliação financeira.
\end{itemize}

\section{Justificativa}
Apresentar a relevância da solução proposta, incluindo dados numéricos e gráficos que demonstrem a extensão e a importância do problema e do impacto da solução.

A importância da solução CortaAi torna-se evidente ao quantificar o custo operacional real da gestão manual para um profissional do setor. A rotina de um gestor ou "Barbeiro Dono" é intensa, totalizando aproximadamente 232 horas de trabalho mensais (considerando uma jornada de 9 horas por dia, 6 dias por semana). O problema central é que uma parcela desproporcional desse tempo é consumida por tarefas administrativas de baixo valor, que poderiam ser facilmente automatizadas.

Conforme ilustrado no Gráfico 1.1 (Redução do Tempo Administrativo com o Sistema), a gestão de agendamentos via métodos desorganizados—como aplicativos de mensagens (WhatsApp) e planilhas manuais—consome, em média, 40 horas por mês do profissional. Este esforço, puramente administrativo, representa 17\% do tempo total de trabalho do barbeiro. Este tempo perdido equivale a quase uma semana inteira de trabalho (4.4 dias) que é desviada da atividade principal e geradora de receita: o atendimento ao cliente. Com a implementação do CortaAi, a plataforma automatiza a maior parte desse esforço, gerando um "Tempo Liberado" de aproximadamente 35 horas mensais (conforme o gráfico). A relevância do sistema reside na capacidade de transformar este tempo improdutivo em oportunidade de novos atendimentos ou gestão estratégica, liberando o profissional para focar no crescimento do negócio.

\begin{figure}[htb]
    \centering
    \includegraphics[width=0.8\textwidth]{images/grafico_reducao_tempo.png}
    \caption{Redução do Tempo Administrativo com o Sistema}
    \label{fig:grafico_tempo}
\end{figure}

Extensão e Importância da Solução: A Recuperação de Clientes
A implementação do sistema de Marketplace para Barbearias se justifica, principalmente, pela sua capacidade de eliminar perdas financeiras e operacionais causadas pelo fenômeno do "no-show" (não comparecimento sem aviso). A taxa de não comparecimento em agendamentos, uma dor generalizada no setor de serviços no Brasil, representa um custo de oportunidade significativo para o Barbeiro. Com uma rotina de aproximadamente 260 agendamentos mensais, o modelo de gestão manual está exposto a uma alta taxa de falha. Conforme ilustrado no Gráfico 1.2: Redução de No-Shows após Implantação do Sistema, o impacto é o seguinte:

\begin{figure}[htb]
    \centering
    \includegraphics[width=0.6\textwidth]{images/grafico_reducao_noshow.png}
    \caption{Redução de No-Shows após Implantação do Sistema}
    \label{fig:grafico_noshow}
\end{figure}

\section{Análise da Concorrência}
Avaliar os concorrentes existentes no mercado, identificando suas forças e fraquezas.

\subsection{Salão99}
O Salão99 (Enterprise) se posiciona como uma suíte de Gestão Empresarial (ERP) para o setor de beleza, oferecendo controle financeiro avançado, gestão de comissões e fluxo de caixa. Sua Proposta de Valor (PV) reside na centralização e na profundidade dos dados gerenciais. No entanto, este escopo amplo implica, muitas vezes, em interfaces mais complexas e recursos que podem ser excessivos ou irrelevantes para as necessidades específicas e simplificadas de uma barbearia de médio ou pequeno porte. O foco em ser "completo" pode comprometer a experiência do usuário (UX) no nicho específico de barbearias.

\subsection{AgendaServiço}
O AgendaServiço adota uma Proposta de Valor de simplicidade e comunicação direta. Seu principal diferencial competitivo é a conexão otimizada via WhatsApp, que reduz a taxa de no-show através de lembretes automáticos e comunicações ágeis. O sistema prioriza a função de link personalizado de agendamento, tornando a organização de horários extremamente acessível. Contudo, essa simplificação tende a limitar a profundidade de ferramentas gerenciais, deixando de lado aspectos cruciais como o controle detalhado de custos e a emissão de relatórios financeiros customizados.

\subsection{eAgenda}
O eAgenda estrutura sua Proposta de Valor em torno da experiência multiplataforma. A existência de um aplicativo (App) dedicado para clientes e profissionais estabelece um diferencial de conveniência e engajamento. O foco está na facilidade de agendamento 24 horas e na gestão visual da agenda. De forma semelhante ao AgendaServiço, o eAgenda se destaca na dimensão Organização, mas é menos explícito ou robusto no módulo de Administração, falhando em oferecer as ferramentas necessárias para uma análise estratégica aprofundada do desempenho e rentabilidade da barbearia (ex: margem de lucro por serviço ou taxa de reincidência por barbeiro). Estes concorrentes, embora resolvam o desafio da organização de agenda, tendem a negligenciar a Gestão Financeira e Operacional interna da barbearia. Esta insuficiência gerencial obriga os gestores a recorrerem a ferramentas externas, como planilhas, fragmentando a administração e resultando em perda de eficiência.

\subsection{Comparativo}
\begin{table}[htb]
    \centering
    \caption{Comparativo de Concorrentes}
    \label{tab:concorrentes}
    \begin{tabular}{lcccc}
        \toprule
        \textbf{Característica} & \textbf{Salão99} & \textbf{AgendaServiço} & \textbf{eAgenda} & \textbf{CortaAi} \\
        \midrule
        Foco Exclusivo (Barbearias) & Não & Não & Não & Sim \\
        Agendamento Online & Sim & Sim & Sim & Sim \\
        Gestão Financeira Completa & Sim & Não & Não & Sim \\
        Comissionamento & Básico & Nulo & Nulo & Específico \\
        Simplicidade da Interface & Média & Alta & Alta & Alta \\
        App Móvel Dedicado & Sim & Não & Sim & Não (Web) \\
        \bottomrule
    \end{tabular}
\end{table}

\chapter{Fundamentação Teórica}
Explicar os tópicos principais do projeto com base em literatura pertinente.
A transição de métodos analógicos para digitais revolucionou a gestão de negócios no setor de serviços, especialmente em estabelecimentos como barbearias e salões de beleza, onde a gestão do tempo e o relacionamento com o cliente são fundamentais para o sucesso.

\section{A Digitalização de serviços e experiência do cliente}
Um estudo da Revista Científica da UNIFENAS (2024) aborda o potencial promissor dos salões de beleza e barbearias, que historicamente enfrentam desafios relacionados à disponibilidade de profissionais, horários e, principalmente, ao tempo de espera dos clientes. Dados coletados por esta pesquisa indicam que 67\% dos clientes expressam o desejo de agendar serviços online, visando otimizar suas agendas ocupadas. A migração de um sistema de agendamento manual, sujeito a erros e desorganização, para uma plataforma digital é apresentada como uma solução direta para essa demanda. A MUTUAL TECNOLOGIA (2025) corrobora essa visão, afirmando que, para a satisfação do cliente, as inovações nos sistemas de agendamento em barbearias são imprescindíveis. A implementação de um sistema eficiente não só simplifica o processo de marcação para o cliente, mas também sinaliza a preocupação do estabelecimento em oferecer um serviço de qualidade, o que resulta em uma experiência de usuário positiva e na redução de cancelamentos de última hora.

\section{Sistemas de gestão e agendamento para barbearias}
O SEBRAE (2023) descreve como softwares e aplicativos especializados facilitam e melhoram a gestão de salões. Essas ferramentas vão além do simples agendamento automatizado, que por si só já permite o controle da agenda e a reserva ou reagendamento de horários. Elas evoluem para sistemas de Gestão de Relacionamento com o Cliente (CRM), permitindo armazenar informações valiosas, como os serviços que o cliente utiliza, a frequência com que frequenta o salão e seu gasto médio. Além disso, o SEBRAE (2023) destaca a integração da agenda com módulos essenciais como "Frente de Caixa", para o controle de pagamentos (cartão, PIX, dinheiro), e "Gestão Financeira", que controla o fluxo de caixa, comissões de barbeiros e a emissão de relatórios estratégicos para a tomada de decisão do proprietário.

Um artigo do IJRASET (2023) define o "Barber Shop Management System" como uma plataforma abrangente que redefine como as barbearias interagem com os clientes e gerenciam seus processos internos. O sistema une as necessidades de proprietários, funcionários e clientes. A plataforma soluciona problemas convencionais como longos tempos de espera e agendamento manual "invisível". Nestes sistemas, os usuários são comumente categorizados em "Barbeiros" (que podem gerenciar seus serviços, visualizar agendamentos e rastrear ganhos) e "Clientes" (que podem se registrar, ver horários disponíveis, escolher serviços com preços e agendar remotamente).

\chapter{Gestão do Projeto}

\section{Organização da Equipe}
A organização da equipe para abranger toda a complexidade do projeto foi pensada no tema que o aluno tem maior desenvoltura, assim agilizando o processo de criação e lapidação do projeto.

\subsection{Responsabilidades / Papéis / Atividades}
Responsabilidades, papéis e atividades de cada membro da equipe.

\begin{table}[htb]
    \centering
    \caption{Responsabilidades da Equipe}
    \label{tab:equipe}
    \begin{tabular}{p{3cm} p{3.5cm} p{7cm}}
        \toprule
        \textbf{Membro da Equipe} & \textbf{Papel} & \textbf{Responsabilidades} \\
        \midrule
        Renan Ladislau & Gerente de Projeto & Planejamento, execução, monitoramento e controle do projeto. \\
        Renan Ladislau & Desenvolvedor Front-end & Implementação da interface do usuário. \\
        Guilherme Chaves/Ana Clara & Desenvolvedor Back-end & Desenvolvimento da lógica de negócios e API. \\
        João Paulo/Laryssa Giovanna & Banco de Dados & Modelagem e gerenciamento do banco de dados. \\
        Guilherme Chaves/Kaique Diniz & Hospedagem & Estudo do serviço de Hospedagem para o projeto (AWS) \\
        \bottomrule
    \end{tabular}
\end{table}

\section{Metodologias de gestão e desenvolvimento}
Como metodologia foi adotado o SCRUM, não de forma arbitrária, mas fundamentada na natureza e contexto de um projeto integrador que por sua vez exige não apenas a entrega de um produto funcional como também a demonstração de aprendizado e adaptabilidade a requisitos emergentes.

Adicionalmente, a estrutura de papéis do Scrum (Proprietário do Produto, Mestre Scrum e Time de Desenvolvimento) e a realização de cerimônias (Planejamento, Reunião Diária, Revisão e Retrospectiva) fomentam a transparência e a colaboração dentro da equipe. Isso é crucial para um Projeto Integrador, pois assegura que todos os membros estejam alinhados com a priorização das funcionalidades e permite a rápida identificação e resolução de impedimentos, potencializando o desempenho do time na busca pela qualidade final do produto.

Portanto, a escolha do Scrum é justificada pela sua eficácia em lidar com a complexidade inerente ao desenvolvimento de software, assegurando um produto que é funcional, adaptável e de valor para o nicho de barbearias.

\section{Repositório da aplicação}
O repositório da aplicação é um ponto-chave para o desenvolvimento e manutenção e para atender as demandas do mercado, juntamente com a capacitação da nossa equipe, decidimos adotar o GitHub como forma de versionamento.

O acesso ao repositório pode ser feito via link, onde não há dificuldades para navegar entre as branches. Em prol do dinamismo no reconhecimento e facilidade em encontrar os arquivos de interesse da pessoa, nomeamos as branches de acordo com o desenvolvimento inerente a ela. Desta forma, ao buscar o backend, deve-se navegar para a branch que representa ele.

Link do repositório: 
\url{https://github.com/AppCortaAi/Arquitetura_completa/tree/main}

\chapter{Desenvolvimento do Projeto}

\section{Escopo do projeto}
O projeto "Cortaai" consiste no desenvolvimento de um sistema de marketplace de barbearias, focado na automação de agendamentos e na gestão de estabelecimentos. O escopo do produto inclui as seguintes funcionalidades, suportadas por um backend robusto em Java/Spring Boot:
\begin{enumerate}
    \item \textbf{Módulo de Autenticação e Usuários (Cliente e Barbeiro):}
        \begin{itemize}
            \item Registro de Clientes e Barbeiros com validação de campos únicos (e-mail, telefone, CPF) e formato (CPF).
            \item Login seguro para ambos os tipos de usuário usando e-mail e senha, com criptografia BCrypt.
            \item Autenticação baseada em token JWT, definindo papéis ( ROLE\_CUSTOMER , ROLE\_BARBER , ROLE\_OWNER ).
            \item Gerenciamento de perfil (atualização de dados cadastrais) para Clientes e Barbeiros autenticados.
        \end{itemize}
    \item \textbf{Módulo de Gerenciamento de Barbearia (Dono):}
        \begin{itemize}
            \item Um Barbeiro pode registrar uma nova barbearia, tornando-se Barbeiro Owner .
            \item Upload de mídia (Logo, Banner, Imagens de Destaque) para o perfil da barbearia, com armazenamento no Cloudinary.
            \item Cadastro de serviços oferecidos pela barbearia, incluindo nome, preço, duração e foto do serviço.
        \end{itemize}
    \item \textbf{Módulo de Gerenciamento de Equipe (Barbeiro e Dono):}
        \begin{itemize}
            \item Barbeiro pode solicitar entrada em uma barbearia existente via CNPJ.
            \item Barbeiro\_Owner pode visualizar, aprovar ou rejeitar solicitações de entrada pendentes de novos barbeiros na barbearia.
            \item Barbeiro pode definir seu horário de trabalho (início e fim).
            \item Barbeiro pode vincular a seu perfil quais serviços da barbearia ele está apto a realizar.
            \item Barbeiro pode consultar seu histórico de solicitações de entrada.
        \end{itemize}
    \item \textbf{Módulo de Agendamento (Core):}
        \begin{itemize}
            \item Cálculo de disponibilidade de horários para um barbeiro, com base em seu horário de trabalho e agendamentos existentes.
            \item Criação e atualização de agendamentos pelo Cliente, com validação de regras de negócio.
            \item Cancelamento de agendamentos pelo Cliente (dono do agendamento) ou pelo Dono da barbearia.
            \item Consulta de agendas (listagem de agendamentos) com visões segregadas por papel.
        \end{itemize}
    \item \textbf{Descoberta}
        \begin{itemize}
            \item Endpoints públicos para listar barbearias, seus serviços e seus barbeiros, permitindo a descoberta pelo cliente.
        \end{itemize}
\end{enumerate}

\subsection*{Premissas}
O sucesso do projeto baseia-se nas seguintes premissas:
\begin{itemize}
    \item \textbf{Infraestrutura Disponível:} Assume-se que a infraestrutura de nuvem (AWS EC2, Amazon RDS) e os serviços de terceiros estarão disponíveis.
    \item \textbf{Credenciais Válidas:} Assume-se que as credenciais de acesso ao banco de dados e serviços externos são válidas.
    \item \textbf{Consumo da API:} Assume-se que uma aplicação frontend (como a prevista em React) será desenvolvida separadamente.
    \item \textbf{Conectividade:} Assume-se que os usuários finais terão acesso à internet.
\end{itemize}

\subsection*{Restrições}
O projeto está sujeito às seguintes restrições:
\begin{itemize}
    \item \textbf{Tecnologia Backend:} O desenvolvimento está restrito ao ecossistema Java 17 e Spring Boot 3.3.4.
    \item \textbf{Persistência de Dados:} O sistema deve utilizar o banco de dados MySQL hospedado no Amazon RDS.
    \item \textbf{Plataforma de Implantação:} A arquitetura de implantação deve será utilizando AWS EC2.
    \item \textbf{Armazenamento de Mídia:} O armazenamento de mídia está restrito ao uso do serviço Cloudinary.
    \item \textbf{Segurança:} O modelo de segurança deve obrigatoriamente implementar autenticação via token JWT.
\end{itemize}

\subsection{Regras de negócio}
Descrever as regras que governam as operações do sistema.
\begin{longtable}{p{1.5cm} p{10.5cm} p{2cm}}
    \caption{Regras de Negócio} \label{tab:rn} \\
    \toprule
    \textbf{Código} & \textbf{Descrição} & \textbf{Relacionados} \\
    \midrule
    \endfirsthead
    \multicolumn{3}{c}{\tablename~\thetable: Continuação} \\
    \toprule
    \textbf{Código} & \textbf{Descrição} & \textbf{Relacionados} \\
    \midrule
    \endhead
    \bottomrule
    \endfoot
    RN1 & "Usuários não autenticados podem acessar endpoints públicos (listagens gerais, registro, login, Swagger)." & \\
    RN2 & "Autenticação é necessária para acessar funcionalidades específicas de Cliente, Barbeiro ou Dono." & \\
    RN3 & "A autenticação é realizada via e-mail e senha, retornando um token JWT." & RN2 \\
    RN4 & O token JWT deve ser enviado no cabeçalho Authorization: Bearer <token> para acessar rotas protegidas. & "RN2, RN3" \\
    RN5 & Apenas o Cliente pode gerenciar seu próprio perfil e agendamentos. & RN2 \\
    RN6 & "Apenas o Barbeiro pode gerenciar seu próprio perfil, solicitar entrada ou sair de uma loja." & RN2 \\
    RN7 & Apenas o Barbeiro Owner pode gerenciar sua barbearia. & "RN2, RN6" \\
    RN8 & Um Barbeiro só pode se tornar Barbeiro Owner ao registrar a sua primeira barbearia. & "RN6, RN7" \\
    RN9 & "E-mail, telefone e CPF devem ser únicos para Clientes e Barbeiros." & \\
    RN10 & CPF e CNPJ devem seguir o formato e algoritmo de validação brasileiros. & \\
    RN11 & "Um agendamento requer um Cliente, Barbeiro, Barbearia e pelo menos um Serviço válido." & \\
    RN12 & O Barbeiro selecionado para o agendamento deve pertencer à Barbearia selecionada. & RN11 \\
    RN13 & Os Serviços selecionados para o agendamento devem pertencer à Barbearia selecionada. & RN11 \\
    RN14 & O Barbeiro selecionado deve estar habilitado a realizar todos os Serviços selecionados. & "RN11, RN13" \\
    RN15 & O horário de início do agendamento deve ser no futuro. & RN11 \\
    RN16 & O horário do agendamento deve estar dentro do horário de trabalho definido para o Barbeiro. & RN11 \\
    RN17 & O horário do agendamento não pode conflitar com outros agendamentos existentes para o mesmo Barbeiro (exceto cancelados). & "RN11, RN16" \\
    RN18 & "Um Cliente ou o Dono da Barbearia podem cancelar um agendamento, desde que não esteja concluído." & "RN5, RN7" \\
    RN19 & Apenas o Dono pode excluir fisicamente um agendamento. & RN7 \\
    RN20 & Um Barbeiro só pode solicitar entrada em uma barbearia se não estiver vinculado a nenhuma. & RN6 \\
    RN21 & Um Dono só pode aprovar uma solicitação se o Barbeiro ainda estiver livre. & "RN7, RN20" \\
    RN22 & O Dono pode rejeitar uma solicitação. & RN7 \\
\end{longtable}

\subsection{Requisitos funcionais}
Listar os requisitos que descrevem as funcionalidades que o sistema deve oferecer.
\begin{longtable}{p{1.5cm} p{10.5cm} p{2cm}}
    \caption{Requisitos Funcionais} \label{tab:rf} \\
    \toprule
    \textbf{Código} & \textbf{Descrição} & \textbf{Relacionados} \\
    \midrule
    \endfirsthead
    \multicolumn{3}{c}{\tablename~\thetable: Continuação} \\
    \toprule
    \textbf{Código} & \textbf{Descrição} & \textbf{Relacionados} \\
    \midrule
    \endhead
    \bottomrule
    \endfoot
    RF1 & O sistema deve permitir o registro de novos Clientes. & "RN9, RN10" \\
    RF2 & O sistema deve permitir o registro de novos Barbeiros. & "RN9, RN10" \\
    RF3 & "O sistema deve permitir o login de Clientes via e-mail e senha, retornando um token JWT." & "RN3, RN4" \\
    RF4 & "O sistema deve permitir o login de Barbeiros via e-mail e senha, retornando um token JWT." & "RN3, RN4" \\
    RF5 & Um Cliente autenticado deve poder visualizar e atualizar seus dados cadastrais (exceto senha). & "RN5, RN9, RN10" \\
    RF6 & Um Barbeiro autenticado deve poder visualizar e atualizar seus dados cadastrais (exceto senha). & "RN6, RN9, RN10" \\
    RF7 & Um Cliente autenticado deve poder fazer upload/atualizar sua foto de perfil. & RN5 \\
    RF8 & Um Barbeiro autenticado deve poder fazer upload/atualizar sua foto de perfil. & RN6 \\
    RF9 & Qualquer usuário (autenticado ou não) deve poder listar todas as Barbearias cadastradas. & RN1 \\
    RF10 & Qualquer usuário deve poder listar os Serviços oferecidos por uma Barbearia específica. & RN1 \\
    RF11 & Qualquer usuário deve poder listar os Barbeiros que trabalham em uma Barbearia específica. & RN1 \\
    RF12 & "Um Barbeiro autenticado (sem loja) deve poder registrar sua primeira Barbearia, tornando-se Dono." & "RN6, RN8, RN10" \\
    RF13 & "Um Dono autenticado deve poder atualizar os dados da sua Barbearia (nome, endereço)." & RN7 \\
    RF14 & Um Dono autenticado deve poder fazer upload/atualizar o logo da sua Barbearia. & RN7 \\
    RF15 & Um Dono autenticado deve poder fazer upload/atualizar o banner da sua Barbearia. & RN7 \\
    RF16 & Um Dono autenticado deve poder adicionar imagens de destaque à sua Barbearia. & RN7 \\
    RF17 & Um Dono autenticado deve poder remover imagens de destaque da sua Barbearia. & RN7 \\
    RF18 & Um Dono autenticado deve poder cadastrar novos Serviços para sua Barbearia. & RN7 \\
    RF19 & Um Dono autenticado deve poder fazer upload/atualizar a foto de um Serviço. & RN7 \\
    RF20 & Um Barbeiro autenticado (sem loja) deve poder solicitar entrada em uma Barbearia existente via CNPJ. & "RN6, RN20" \\
    RF21 & Um Dono autenticado deve poder visualizar as solicitações de entrada pendentes para sua Barbearia. & RN7 \\
    RF22 & Um Dono autenticado deve poder aprovar uma solicitação de entrada. & "RN7, RN21" \\
    RF23 & Um Dono autenticado deve poder rejeitar uma solicitação de entrada. & "RN7, RN22" \\
    RF24 & Um Barbeiro autenticado (não-dono) deve poder sair da Barbearia à qual está vinculado. & RN6 \\
    RF25 & Um Dono autenticado deve poder remover um Barbeiro (não-dono) da sua Barbearia. & RN7 \\
    RF26 & Um Barbeiro autenticado deve poder definir seu horário de trabalho. & RN6 \\
    RF27 & Um Barbeiro autenticado (vinculado a uma loja) deve poder definir quais Serviços da loja ele realiza. & RN6 \\
    RF28 & Qualquer usuário deve poder consultar os horários disponíveis de um Barbeiro para uma data e duração específicas. & "RN1, RN16, RN17" \\
    RF29 & Um Cliente autenticado deve poder criar um novo Agendamento. & "RN5, RN11, RN12, RN13, RN14, RN15, RN16, RN17" \\
    RF30 & Um Cliente autenticado deve poder atualizar um Agendamento existente (não concluído/cancelado). & "RN5, RN11-RN17" \\
    RF31 & Um Cliente autenticado deve poder cancelar um Agendamento seu (não concluído). & "RN5, RN18" \\
    RF32 & Um Dono autenticado deve poder cancelar qualquer Agendamento da sua loja (não concluído). & "RN7, RN18" \\
    RF33 & Um Dono autenticado deve poder excluir fisicamente um Agendamento da sua loja. & "RN7, RN19" \\
    RF34 & Um Cliente autenticado deve poder consultar a lista dos seus Agendamentos. & RN5 \\
    RF35 & Um Barbeiro autenticado deve poder consultar a lista dos seus Agendamentos. & RN6 \\
    RF36 & Um Dono autenticado deve poder consultar a lista de todos os Agendamentos da sua loja. & RN7 \\
    RF37 & Um Barbeiro autenticado deve poder consultar o histórico das suas solicitações de entrada. & RN6 \\
    RF38 & O sistema deve prover documentação da API via Swagger UI. & RN1 \\
\end{longtable}

\subsection{Requisitos não funcionais}
Listar os requisitos que descrevem as características de qualidade do sistema.
\begin{itemize}
    \item \textbf{Escalabilidade:} O sistema deve ser capaz de escalonar automaticamente utilizando AWS Lambda e RDS.
    \item \textbf{Segurança:} A aplicação deve garantir a autenticação via JWT e autorização baseada em roles. Senhas devem ser armazenadas com hash (BCrypt). Dados sensíveis em trânsito devem usar HTTPS.
    \item \textbf{Usabilidade e Acessibilidade:} A interface do frontend (React - previsto) deve ser intuitiva e responsiva.
    \item \textbf{Desempenho:} A consulta de disponibilidade de horários (getAvailableSlots) deve ter tempo de resposta otimizado.
    \item \textbf{Disponibilidade:} A arquitetura serverless na AWS (Lambda, RDS) visa alta disponibilidade (24/7).
    \item \textbf{Manutenibilidade:} O código Java deve seguir convenções de estilo, ser modularizado (Controller/Service/Repository) e utilizar JPA e DTOs (MapStruct).
    \item \textbf{Registro de Logs:} A aplicação deve utilizar SLF4J/Logback para registrar eventos.
\end{itemize}

\section{Histórias de usuário}
As Histórias de Usuário foram definidas com base nos casos de uso para os três atores principais do sistema: Cliente, Barbeiro e Barbeiro Dono.

\subsection{Cadastro e agendamento (Cliente)}
\textbf{Eu como} um novo usuário, \textbf{quero} me registrar como Cliente e realizar meu primeiro agendamento, \textbf{para} poder utilizar os serviços de barbearia disponíveis na plataforma.
\begin{description}
    \item[Critérios de Aceitação:]
    \item[Critério 1:] O usuário deve preencher o formulário de registro com dados válidos.
    \item[Critério 2:] O sistema deve validar a unicidade do e-mail, telefone e CPF.
    \item[Critério 3:] Após o registro, o usuário deve poder realizar login para obter um token JWT.
    \item[Critério 4:] Com o token JWT, o Cliente deve poder listar barbearias, serviços, barbeiros e consultar disponibilidade.
    \item[Critério 5:] O Cliente deve poder criar um novo agendamento.
    \item[Critério 6:] O sistema deve validar todas as regras de negócio do agendamento.
\end{description}

\subsection{Gerenciar Perfil e Agenda (Barbeiro)}
\textbf{Eu como} Barbeiro cadastrado, \textbf{quero} gerenciar meu perfil, definir meus serviços, meu horário de trabalho e consultar minha agenda, \textbf{para} manter minhas informações atualizadas e organizar meu dia de trabalho.
\begin{description}
    \item[Critérios de Aceitação:]
    \item[Critério 1:] O Barbeiro deve poder fazer login para obter um token JWT.
    \item[Critério 2:] O Barbeiro autenticado deve poder atualizar seus dados pessoais e foto.
    \item[Critério 4:] O Barbeiro autenticado (vinculado a uma loja) deve poder definir quais serviços da sua loja ele realiza.
    \item[Critério 5:] O Barbeiro autenticado deve poder definir seu horário de início e fim de expediente.
    \item[Critério 6:] O Barbeiro autenticado deve poder consultar sua própria lista de agendamentos.
\end{description}

\subsection{Gerenciar Barbearia (Barbeiro Dono)}
\textbf{Eu como} Barbeiro Dono, \textbf{quero} gerenciar os dados da minha barbearia, o menu de serviços, a equipe de barbeiros e as solicitações de entrada, \textbf{para} administrar completamente meu estabelecimento na plataforma.
\begin{description}
    \item[Critérios de Aceitação:]
    \item[Critério 1:] O usuário deve estar autenticado como Barbeiro Dono.
    \item[Critério 2:] O Dono deve poder atualizar os dados da sua barbearia.
    \item[Critério 3:] O Dono deve poder gerenciar as imagens da barbearia (logo, banner, destaques).
    \item[Critério 4:] O Dono deve poder cadastrar novos serviços.
    \item[Critério 5:] O Dono deve poder visualizar as solicitações pendentes.
    \item[Critério 6:] O Dono deve poder aprovar ou rejeitar uma solicitação de entrada.
    \item[Critério 7:] O Dono deve poder remover um barbeiro (não-dono) da sua equipe.
    \item[Critério 8:] O Dono deve poder consultar a agenda completa da sua barbearia.
\end{description}

\section{Arquitetura da aplicação}
A arquitetura do sistema adota um modelo Cliente-Servidor distribuído. O frontend (cliente), previsto em React, é responsável pela interface do usuário e consome a API REST exposta pelo backend (servidor), que será desenvolvido em Java com Spring Boot.

O backend é estruturado utilizando uma arquitetura em camadas (layered architecture), um padrão consolidado no ecossistema Spring, que divide as responsabilidades da seguinte forma:
\begin{itemize}
    \item \textbf{Controllers:} Camada de entrada da API REST. É responsável por receber as requisições HTTP, validar os dados de entrada (DTOs) e delegar a lógica de negócios para a camada de serviço.
    \item \textbf{Services:} Camada que contém a lógica de negócios principal da aplicação. Executa as regras de agendamento, validações complexas e orquestra as operações.
    \item \textbf{Repositories:} Camada de acesso aos dados. Utiliza o Spring Data JPA para abstrair as operações de persistência (CRUD) com o banco de dados relacional (MySQL).
    \item \textbf{Models (Entities):} Representam as tabelas do banco de dados como objetos Java (POJOs), utilizando anotações da JPA.
    \item \textbf{Mappers:} Componentes responsáveis por converter eficientemente os objetos de Entidade em DTOs, utilizando a biblioteca MapStruct.
\end{itemize}

\begin{figure}[htb]
    \centering
    \includegraphics[width=\textwidth]{images/diagrama_arquitetura.png}
    \caption{Diagrama de Componentes da Arquitetura}
    \label{fig:arquitetura}
\end{figure}

\section{Tecnologias}
Esta seção descreve a pilha de tecnologia (tech stack) principal para o projeto juntamente de sua arquitetura.

\subsection{Front-end}
\begin{itemize}
    \item \textbf{React:} Biblioteca JavaScript prevista para o desenvolvimento do frontend.
    \item \textbf{Axios:} Biblioteca JavaScript prevista para realizar requisições HTTP.
\end{itemize}

\subsection{Back-end}
\begin{itemize}
    \item \textbf{Java 17:} Linguagem principal do back-end.
    \item \textbf{Spring Boot 3.3.4:} Framework principal, simplificando a configuração.
    \item \textbf{Spring Data JPA / Hibernate:} Camada de persistência de dados (ORM).
    \item \textbf{Spring Security:} Framework para gerenciamento de autenticação e autorização.
    \item \textbf{JWT (Java Web Token):} Utilizado para a autenticação baseada em tokens stateless.
    \item \textbf{MapStruct:} Ferramenta para geração de código de mapeamento (Entidades e DTOs).
    \item \textbf{Jakarta Bean Validation:} Especificação para validação de dados em DTOs.
    \item \textbf{SLF4J / Logback:} Facade e implementação de logging.
\end{itemize}

\subsection{Banco de dados}
\begin{itemize}
    \item \textbf{MySQL:} Sistema de gerenciamento de banco de dados relacional.
\end{itemize}

\subsection{Infraestrutura}
A infraestrutura do projeto é baseada em uma arquitetura de nuvem serverless:
\begin{itemize}
    \item \textbf{Amazon Web Services (AWS):} Provedor de nuvem principal.
    \item \textbf{AWS Lambda (Computação Serverless):} O backend Java Spring Boot é empacotado para rodar como uma função Lambda.
    \item \textbf{Amazon RDS (Relational Database Service):} O banco de dados MySQL é hospedado como um serviço gerenciado.
    \item \textbf{Amazon API Gateway (Implícito):} Atua como o ponto de entrada da aplicação, recebendo as requisições HTTP.
    \item \textbf{Cloudinary (Plataforma de Mídia SaaS):} Serviço de terceiros especializado em gerenciamento de mídia (upload, armazenamento e entrega de imagens).
\end{itemize}

\subsection{Ferramentas de Apoio}
\begin{itemize}
    \item \textbf{Trello (Gestão de Projetos e Tarefas):} Ferramenta central para o gerenciamento de tarefas e acompanhamento do progresso (Kanban).
    \item \textbf{GitHub (Controle de Versão e Repositório de Código):} Sistema de controle de versão (VCS) e repositório central de todo o código-fonte.
    \item \textbf{Overleaf (Documentação Colaborativa):} Plataforma adotada para a elaboração e manutenção da documentação oficial do projeto (LaTeX).
    \item \textbf{Discord (Comunicação da Equipe):} Plataforma para a comunicação síncrona (instantânea) da equipe.
\end{itemize}

\section{Testes e manutenibilidade}
O Plano de Testes serve como um mapa ou guia para todo o esforço de teste do projeto. Em essência, ele responde às seguintes perguntas:
\begin{itemize}
    \item \textbf{Por que} estamos testando? (Definido nos Objetivos)
    \item \textbf{O que} exatamente vamos testar? (Definido no Escopo de Teste)
    \item \textbf{Como} vamos testar? (Definido na Estratégia de Teste)
    \item \textbf{Com o que} vamos testar? (Definido nas Ferramentas de Teste)
    \item \textbf{Onde} vamos testar? (Definido nos Ambientes de Teste)
    \item \textbf{Quando} terminamos? (Definido nos Critérios de Conclusão)
\end{itemize}

\subsection{Estratégia e Escopo de Teste}
Foi adotado uma estratégia que segue a pirâmide de testes. O escopo de teste é definido dentro de cada nível estratégico.

\subsubsection{Nível 1: Testes Unitários}
\begin{itemize}
    \item \textbf{Escopo e Foco:} Lógica de negócio. O escopo inclui todas as classes de Serviço (pacote \texttt{service.impl}).
    \item \textbf{Descrição e Ferramentas:} Cada método de serviço (ex: \texttt{getAvailableSlots}) é testado de forma isolada. Dependências são "mockadas" utilizando Mockito.
    \item \textbf{Execução:} Automaticamente a cada build local (\texttt{mvn test}).
\end{itemize}

\subsubsection{Nível 2: Testes de Integração}
\begin{itemize}
    \item \textbf{Escopo e Foco:} Fluxo completo da API (Controller -> Service -> BD) e a camada de segurança.
    \item \textbf{Descrição e Ferramentas:} Utilizando \texttt{@SpringBootTest} e \texttt{MockMvc}, estes testes validam o fluxo CRUD, a lógica de segurança (JWT 401/403) e validações de DTO (400 Bad Request).
    \item \textbf{Execução:} Automaticamente a cada build do Maven e no pipeline de CI/CD.
\end{itemize}

\subsubsection{Nível 3: Testes de Componente/API}
\begin{itemize}
    \item \textbf{Escopo e Foco:} Validação do contrato da API (JSON) e a integração real com serviços externos (Cloudinary e AWS RDS) em ambiente de homologação (Staging).
    \item \textbf{Descrição e Ferramentas:} Após o deploy em Staging, são executados testes manuais (Insomnia ou Postman) para verificar a integração real.
    \item \textbf{Execução:} Manual, realizado antes de qualquer promoção do código para Produção.
\end{itemize}

\subsection{Ambientes e Ferramentas de Teste}
\begin{table}[htb]
    \centering
    \caption{Ambientes e Ferramentas de Teste}
    \label{tab:ambientes}
    \resizebox{\textwidth}{!}{%
    \begin{tabular}{p{3cm} p{5.5cm} p{5cm} p{5cm}}
        \toprule
        \textbf{Ambiente} & \textbf{Propósito e Tipos de Teste} & \textbf{Infraestrutura e Banco de Dados} & \textbf{Ferramentas Aplicáveis} \\
        \midrule
        \textbf{Local} & Desenvolvimento, depuração e execução de testes automatizados rápidos. \newline \textbf{Testes:} Unitários e Integração. & Máquina do Desenvolvedor \newline \textbf{BD:} H2 (Em memória) & JUnit 5, Mockito, Spring Boot Test, MockMvc \\
        \textbf{Staging (Homologação)} & Simulação do ambiente de produção para validação final. \newline \textbf{Testes:} Componente/API e Não Funcionais (Carga/Estresse). & AWS Lambda + API Gateway + Cloudinary (Conta de Teste) \newline \textbf{BD:} AWS RDS (Instância de Teste) & Insomnia/Postman, JMeter/Gatling, AWS CloudWatch \\
        \textbf{Produção} & Ambiente final do usuário. \newline \textbf{Testes:} Monitoramento e "Smoke Tests". & AWS Lambda + API Gateway + Cloudinary (Produção) \newline \textbf{BD:} AWS RDS (Produção) & AWS CloudWatch \\
        \bottomrule
    \end{tabular}
    }
\end{table}

\subsection{Testes funcionais}
A estratégia de testes funcionais visa validar o software em diferentes níveis de granularidade.

\subsubsection{Testes Unitários}
\textbf{Definição:} Focam em validar a menor unidade de lógica da aplicação de forma isolada (classes de Serviço). \textbf{Implementação:} Utiliza-se JUnit 5 e Mockito para simular dependências.
\begin{description}
    \item[Cenário de Exemplo (Teste Unitário para \texttt{BarberServiceImpl.getAvailableSlots}):]
    \item[Dado:] Um Barber com horário das 09:00 às 18:00.
    \item[E (And):] O AppointmentsRepository (mockado) retorna um agendamento das 10:00 às 11:00.
    \item[Quando (When):] O método \texttt{getAvailableSlots} é invocado para 60 minutos.
    \item[Então (Then):] O teste afirma que a lista de horários retornada contém "09:00" e "11:00", mas não contém "10:00".
\end{description}

\subsubsection{Testes de Componente}
\textbf{Definição:} Validam o comportamento de um componente ou um pequeno grupo de componentes.
\begin{itemize}
    \item \textbf{Camada de Dados:} Utiliza-se \texttt{@DataJpaTest} do Spring Boot para validar a camada de persistência (JPA/Repositórios) contra um banco H2. Foco em consultas customizadas.
    \item \textbf{Contrato de API:} Utiliza-se Insomnia ou Postman para executar requisições HTTP contra a aplicação e validar os códigos de status (200, 400, 403) e a estrutura do JSON.
\end{itemize}

\subsubsection{Testes de Integração}
\textbf{Definição:} Pilar de validação do backend. Verificam a colaboração entre as camadas (Controller, Service, Repository) e a integração com o banco de dados.
\textbf{Implementação:} Utiliza-se \texttt{@SpringBootTest} para carregar o contexto completo e \texttt{MockMvc} para simular requisições HTTP.
\begin{description}
    \item[Cenário de Exemplo (POST /api/appointments):]
    \item[Dado:] Um banco H2 pré-populado e um usuário simulado \texttt{@WithMockUser} (ROLE\_CUSTOMER).
    \item[Quando (When):] MockMvc executa um POST para /api/appointments com um DTO válido.
    \item[Então (Then):] O teste afirma que a resposta HTTP é 201 (Created) e que o AppointmentsRepository.count() é 1.
    \item[Cenário de Exemplo (Acesso Negado):]
    \item[Dado:] Um usuário simulado \texttt{@WithMockUser} (ROLE\_CUSTOMER).
    \item[Quando (When):] MockMvc executa um GET para uma rota de Dono (ex: /api/barbershops/my-shop/pending-requests).
    \item[Então (Then):] O teste afirma que a resposta HTTP é 403 (Forbidden).
\end{description}

\subsubsection{Testes end-to-end}
\textbf{Definição:} Simulam a jornada completa do usuário real no sistema, validando a integração entre o frontend (React) e o backend (Spring Boot) juntos.
\textbf{Implementação:} Requer que ambas as aplicações estejam implantadas em Staging. Utiliza-se ferramentas de automação de navegador (Cypress ou Playwright).
\begin{description}
    \item[Cenário de Exemplo (Jornada de Agendamento):]
    \item[Setup:] O banco de dados de teste é populado.
    \item[Login:] O script abre o navegador, insere e-mail e senha de um Cliente e clica em "Entrar".
    \item[Navegação:] O script clica na "Barbearia X", depois "Barbeiro Y".
    \item[Seleção:] O script seleciona o "Serviço Z" e clica em um horário disponível ("14:00").
    \item[Ação:] O script clica em "Confirmar Agendamento".
    \item[Verificação:] O script afirma que o agendamento aparece na lista de "Meus Agendamentos".
\end{description}

\subsection{Testes não funcionais}
Verificam *como* o sistema opera sob determinadas condições, focando em desempenho, confiabilidade e configuração.

\subsubsection{Testes de performance}
\textbf{Definição:} Medem a velocidade, responsividade e estabilidade da API sob carga.
\begin{itemize}
    \item \textbf{Testes de Estresse (Stress Testing):} Aumenta progressivamente a carga além do esperado, até o sistema falhar, identificando o ponto de ruptura (limites de concorrência do Lambda, pool de conexões do RDS).
    \item \textbf{Testes de "Cold Start" (Inicialização a Frio):} Específico para AWS Lambda. Mede o tempo que a aplicação Spring Boot leva para inicializar do zero após um período de inatividade.
\end{itemize}

\subsubsection{Testes de configuração}
\textbf{Definição:} Validam que a aplicação funciona corretamente em seu ambiente de implantação final (AWS).
\begin{itemize}
    \item \textbf{Validação de Variáveis de Ambiente:} Verifica se a aplicação consegue ler e utilizar as variáveis essenciais (JDBC\_DATABASE\_URL, CLOUDINARY\_URL, JWT\_SECRET\_KEY).
    \item \textbf{Validação de Conectividade de Rede (AWS):} Testa se as regras (Security Groups) permitem que o Lambda acesse o RDS e o Cloudinary.
    \item \textbf{Validação de Configuração da Aplicação:} Testa se as configurações do \texttt{application.yml} (ex: \texttt{app.availability.slot-interval-minutes: 15}) estão sendo aplicadas.
    \item \textbf{Validação de Configuração de Segurança:} Testa se as regras do \texttt{SecurityConfig} (restrição de rotas) estão sendo aplicadas no ambiente.
\end{itemize}

\subsection{Testes automatizados}
Para a geração de arquivos de logs no backend (Java/Spring Boot), são utilizadas as ferramentas padrão SLF4J (fachada) e Logback (implementação). Os níveis de log utilizados seguem a semântica padrão:
\begin{itemize}
    \item \textbf{ERROR:} Falhas críticas (ex: falha ao conectar no banco).
    \item \textbf{WARN:} Eventos inesperados (ex: tentativa de cancelar agendamento já cancelado).
    \item \textbf{INFO:} Mensagens informativas de progresso (ex: confirmação de criação/atualização).
    \item \textbf{DEBUG:} Mensagens detalhadas para desenvolvimento (ex: visualização de queries SQL).
    \item \textbf{TRACE:} Nível mais detalhado (ex: valores de parâmetros de queries).
\end{itemize}

\subsection{Análise estática}
Visto a importância de se preservar a qualidade e segurança do código, é essencial realizar a Análise Estática de Código e Segurança (SAST). Para isso, empregaremos a ferramenta SonarQube. O SonarQube é uma plataforma de código aberto para inspeção contínua da qualidade do código, integrada ao processo de build (Maven). Ele analisa o código-fonte Java em busca de:
\begin{itemize}
    \item Bugs: Erros potenciais na lógica.
    \item Vulnerabilidades de Segurança: Falhas que podem ser exploradas.
    \item Code Smells: Problemas de design ou manutenibilidade.
    \item Duplicação de Código: Trechos de código repetidos.
    \item Cobertura de Testes: Verifica a porcentagem do código coberta pelos testes unitários (JUnit).
\end{itemize}

\subsection{Code convention}
Para garantir a qualidade, legibilidade e manutenibilidade do código-fonte, foram adotadas convenções de codificação padrão.

\subsubsection{Convenções de Backend (Java / Spring Boot)}
A convenção de código adotada é o \textbf{Google Java Style Guide}, complementado por práticas comuns do Spring Boot.
\begin{itemize}
    \item \textbf{Nomenclatura:}
        \begin{itemize}
            \item Classes e Interfaces: \texttt{UpperCamelCase}.
            \item Métodos e Variáveis: \texttt{lowerCamelCase}.
            \item Constantes: \texttt{CONSTANT\_CASE}.
            \item Pacotes: \texttt{lowercase} (ex: \texttt{ifsp.edu.projeto.cortaai.service.impl}).
        \end{itemize}
    \item \textbf{Formatação:} Indentação de 4 espaços e organização padrão de \texttt{import}.
    \item \textbf{Práticas:} Uso extensivo de anotações do Spring (@Service, @RestController) e JPA (@Entity, @Id), com comentários no padrão Javadoc.
\end{itemize}

\subsubsection{Convenções de Frontend (React)}
A convenção a ser adotada será o \textbf{Airbnb JavaScript Style Guide}, incluindo suas extensões para React.
\begin{itemize}
    \item \textbf{Nomenclatura:}
        \begin{itemize}
            \item Componentes React: \texttt{UpperCamelCase} (ex: \texttt{AppointmentForm}).
            \item Variáveis, Funções e Props: \texttt{lowerCamelCase}.
        \end{itemize}
    \item \textbf{Práticas:} Preferência por componentes funcionais com Hooks, uso de \texttt{const} sobre \texttt{let}, e *arrow functions*.
    \item \textbf{Formatação:} A consistência do estilo será garantida automaticamente por ferramentas como \textbf{Prettier} e \textbf{ESLint}.
\end{itemize}

\section{Segurança, privacidade e legislação}
Para garantir a segurança dos dados e a privacidade dos usuários, os seguintes critérios foram implementados:
\begin{itemize}
    \item \textbf{Autenticação Robusta:} Utiliza Spring Security. Senhas são armazenadas com \texttt{BCryptPasswordEncoder}. O acesso às funcionalidades privadas é feito via JSON Web Tokens (JWT).
    \item \textbf{Autorização Baseada em Papéis:} A arquitetura define \textit{roles} específicas (Cliente, Barbeiro, Barbeiro\_Owner) gerenciadas pelo Spring Security, implementando o princípio do menor privilégio.
    \item \textbf{Validação de Entrada de Dados:} Jakarta Bean Validation é aplicado nos DTOs para validar os dados recebidos, incluindo validações customizadas para unicidade (e-mail, CPF) e formato (CPF, CNPJ).
    \item \textbf{Segurança na Infraestrutura:} O deployment na AWS Lambda e o acesso a serviços externos (Cloudinary, RDS) é protegido por credenciais seguras gerenciadas como variáveis de ambiente.
    \item \textbf{Direitos do Usuário:} O sistema garante aos usuários direitos de acesso, correção e exclusão de seus dados pessoais.
\end{itemize}

\subsection{Observância à Legislação}
O projeto foi desenvolvido com atenção à conformidade com a Lei Geral de Proteção de Dados (LGPD - Lei nº 13.709/2018):
\begin{itemize}
    \item \textbf{Finalidade Específica e Consentimento:} A LGPD exige que dados pessoais (nome, telefone, e-mail, CPF) sejam coletados com consentimento explícito e utilizados apenas para a finalidade informada (agendamentos, gerenciamento).
    \item \textbf{Proteção de Dados Pessoais:} Medidas como criptografia de senhas, controle de acesso e validação de dados são implementadas para proteger os dados.
    \item \textbf{Direitos dos Titulares:} A LGPD garante aos titulares (usuários) direitos como acesso, correção, eliminação e portabilidade de seus dados.
    \item \textbf{Transparência:} A política de privacidade da aplicação deve informar claramente quais dados são coletados, como são utilizados e por quanto tempo são armazenados.
\end{itemize}

\section{Modelo de banco de dados}
Esta seção apresenta a estrutura de dados projetada para suportar a aplicação. O modelo de banco de dados é o alicerce do backend, sendo fundamental para garantir a integridade, consistência e performance.

\subsection{Modelo entidade relacionamento - MER}
O Modelo Entidade-Relacionamento (MER) é o modelo conceitual que identifica as principais entidades de negócio do sistema.

\begin{figure}[htb]
    \centering
    \includegraphics[width=\textwidth]{images/mer.png}
    \caption{Modelo Entidade-Relacionamento (MER)}
    \label{fig:mer}
\end{figure}

\subsection{Diagrama entidade relacionamento - DER}
O Diagrama Entidade-Relacionamento (DER) é a representação gráfica e visual do modelo conceitual (MER). Ele traduz as entidades e relacionamentos em um esquema que se aproxima do modelo lógico do banco de dados.

\begin{figure}[htb]
    \centering
    \includegraphics[width=\textwidth]{images/der.png}
    \caption{Diagrama Entidade-Relacionamento (DER) - Modelo Lógico/Físico}
    \label{fig:der}
\end{figure}

As principais entidades (tabelas) que compõem o sistema são:
\begin{description}
    \item[BARBERSHOPS (Barbearias):] Contém os dados cadastrais da barbearia (nome, CNPJ, endereço).
    \item[CUSTOMERS (Clientes):] Armazena as informações dos clientes (dados pessoais, telefone, e-mail, CPF, senha).
    \item[BARBERS (Barbeiros):] Contém os dados dos profissionais (nome, telefone, e-mail, CPF, senha). Inclui \texttt{is\_owner}, horários de trabalho e \texttt{barbershop\_id}.
    \item[ACTIVITIES (Atividades/Serviços):] Representa os serviços oferecidos (nome, preço, duração) e \texttt{barbershop\_id}.
    \item[APPOINTMENTS (Agendamentos):] Entidade central. Registra os horários (\texttt{start\_time}, \texttt{end\_time}, \texttt{status}) e chaves estrangeiras (customer\_id, barber\_id, barbershop\_id).
    \item[BARBERSHOP\_JOIN\_REQUESTS:] Tabela de controle para gerenciar as solicitações de barbeiros para se juntarem a uma barbearia.
\end{description}

\subsubsection{Tabelas de Junção ( M:N )}
\begin{description}
    \item[BARBER\_ACTIVITIES:] Relaciona quais atividades (\texttt{activity\_id}) um barbeiro (\texttt{barber\_id}) está habilitado a realizar.
    \item[APPOINTMENT\_ACTIVITIES:] Especifica quais atividades (\texttt{activity\_id}) estão incluídas em um agendamento (\texttt{appointment\_id}).
\end{description}

\subsubsection{Relacionamentos Chave ( Cardinalidade )}
\begin{itemize}
    \item \textbf{Barbearia e Barbeiro (Emprega):} Uma barbearia (BARBERSHOPS) emprega muitos barbeiros (BARBERS) (1:N).
    \item \textbf{Barbearia e Atividade (Oferece):} Uma barbearia (BARBERSHOPS) oferece múltiplas atividades (ACTIVITIES) (1:N).
    \item \textbf{Cliente e Agendamento (Realiza):} Um cliente (CUSTOMERS) realiza vários agendamentos (APPOINTMENTS) (1:N).
    \item \textbf{Barbeiro e Agendamento (Atende):} Um barbeiro (BARBERS) pode atender a muitos agendamentos (APPOINTMENTS) (1:N).
    \item \textbf{Barbeiro e Atividade (É realizada por):} O relacionamento entre BARBERS e ACTIVITIES é de muitos para muitos (M:N), mediado pela tabela BARBER\_ACTIVITIES.
    \item \textbf{Agendamento e Atividade (Inclui):} O relacionamento entre APPOINTMENTS e ACTIVITIES é de muitos para muitos (M:N), resolvido pela tabela APPOINTMENT\_ACTIVITIES.
\end{itemize}

\section{Cronograma do projeto}
A execução do projeto seguirá o framework ágil Scrum. O cronograma não é baseado em fases sequenciais, mas em uma série de ciclos iterativos (Sprints).
\begin{itemize}
    \item \textbf{Cadência das Sprints:} 8 Sprints de 2 semanas (10 dias úteis) cada.
    \item \textbf{Product Backlog:}
        \begin{itemize}
            \item Épico 1: Gestão de Acesso (Autenticação JWT, Perfis).
            \item Épico 2: Módulo de Agendamento (Visualização, criação/cancelamento).
            \item Épico 3: Módulo de Serviços e Inventário (Cadastro, preços, duração).
            \item Épico 4: Painel Administrativo (Gestão de agenda e serviços).
            \item Épico 5: Infraestrutura e Implantação (Configuração AWS Lambda, RDS, CI/CD).
        \end{itemize}
\end{itemize}

\subsection{Cronograma de Sprints}
A linha do tempo do projeto é dividida em 8 ciclos curtos de 2 semanas cada. Cada Sprint tem uma Meta clara, focada em entregar uma parte funcional do produto.

\begin{table}[htb]
    \centering
    \caption{Cronograma de Sprints}
    \label{tab:sprints}
    \resizebox{\textwidth}{!}{%
    \begin{tabular}{p{1.5cm} p{3.5cm} p{8.5cm} p{5.5cm}}
        \toprule
        \textbf{Sprint} & \textbf{Período (Início/Fim)} & \textbf{Meta da Sprint (Sprint Goal)} & \textbf{Marcos Equivalentes (Plano Antigo)} \\
        \midrule
        Sprint 1 & 15/08/2025 - 29/08/2025 & "Setup e Fundação: Configurar ambientes (Git, CI/CD, Trello), definir o Product Backlog (MVP) e criar o ""Walking Skeleton""." & M1: Requisitos e Escopo Aprovados \\
        Sprint 2 & 01/09/2025 - 12/09/2025 & Autenticação: Implementar o fluxo de autenticação (JWT) e o CRUD básico de Usuários. Definir a arquitetura de base. & M2: Arquitetura e Design Concluídos \\
        Sprint 3 & 15/09/2025 - 26/09/2025 & MVP do Agendamento (Back-end): API para lógica de horários disponíveis e criação de agendamento. & - \\
        Sprint 4 & 29/09/2025 - 10/10/2025 & MVP de Serviços (Back-end): APIs para CRUD de Serviços e Inventário. & M3: Desenvolvimento Back-end Concluído \\
        Sprint 5 & 13/10/2025 - 24/10/2025 & MVP do Cliente (Front-end): Telas de Login e fluxo principal do cliente (ver horários e realizar agendamento). & - \\
        Sprint 6 & 27/10/2025 - 07/11/2025 & Painel do Barbeiro (Front-end): Telas para o barbeiro visualizar e gerenciar sua agenda e serviços. & M4: Desenvolvimento Front-end Concluído \\
        Sprint 7 & 10/11/2025 - 21/11/2025 & "Testes de Aceite (UAT): Foco em testes de integração (E2E), correção de bugs e validação do fluxo com usuários-piloto." & M5: Ciclo de Testes (UAT) Finalizado \\
        Sprint 8 & 24/11/2025 - 05/12/2025 & "Hardening e Implantação: Preparação do ambiente de produção (AWS), testes, deploy e documentação final." & M6: Implantação (Go-Live) \\
        \bottomrule
    \end{tabular}
    }
\end{table}

\chapter{Considerações Finais}
O desenvolvimento da aplicação web "Cortaai", um marketplace para gerenciamento e agendamento em barbearias, representa uma iniciativa significativa para modernizar e otimizar a operação destes estabelecimentos, trazendo organização, eficiência e maior visibilidade no ambiente digital.
Ao adotar uma arquitetura robusta e escalável baseada em serviços na nuvem (Java/Spring Boot backend hospedado na AWS Lambda, com frontend previsto em React e banco de dados MySQL no RDS), foi possível estruturar uma solução que atende às complexidades do agendamento e da gestão de múltiplos usuários e estabelecimentos.

Ao longo do projeto, foi fundamental analisar e implementar funcionalidades que atendessem às necessidades dos diferentes atores do sistema: Clientes buscando conveniência para encontrar barbearias e agendar serviços online; Barbeiros necessitando de uma ferramenta para gerenciar seu perfil, habilidades, horários e agenda; e Donos de barbearias buscando controle sobre o cadastro da loja, menu de serviços, equipe e fluxo de agendamentos. A implementação da autenticação via JWT e autorização baseada em papéis garante a segurança e o acesso adequado às funcionalidades.

A aplicação "Cortaai" não só permite aos Clientes realizar agendamentos de forma eficiente, verificando a disponibilidade em tempo real, mas também oferece aos Barbeiros e Donos ferramentas para facilitar o controle da agenda, a gestão de serviços, da equipe e da imagem do estabelecimento (upload de logo, banner, destaques via Cloudinary). Espera-se que a plataforma resulte em uma melhoria significativa na experiência dos clientes e na otimização dos processos de gestão das barbearias.

Além disso, a arquitetura implementada e as tecnologias utilizadas fornecem uma base sólida e escalável para iniciativas futuras, como a integração de sistemas de pagamento online, o desenvolvimento de funcionalidades avançadas de gestão de relacionamento com o cliente (CRM) e a implementação de painéis com análises e insights de negócio.

Em suma, o desenvolvimento desta aplicação representa um passo importante em direção à transformação digital e ao aumento da competitividade no setor de barbearias, demonstrando o potencial das tecnologias web modernas e da computação em nuvem para resolver desafios operacionais cotidianos.

% --- ELEMENTOS PÓS-TEXTUAIS ---
\postextual

% --- REFERÊNCIAS ---
\begin{thebibliography}{99}
    \bibitem{carvalho2025}
    CARVALHO, Marcella Souto; TOZI, Luiz Antonio. \textit{Uso de tecnologia como apoio a atividades de gestão em pequenos negócios: caso da barbearia}. Acesso em: 29/08/2025.

    \bibitem{cloudinary2024}
    CLOUDINARY. \textit{Java Integration}. Santa Clara, CA, [c2024]. Disponível em: \url{https://cloudinary.com/documentation/java_integration}. Acesso em: 29/08/2025.

    \bibitem{idealsales2025}
    IDEAL SALES. \textit{Como O CRM Pode Fidelizar Clientes E Aumentar O Faturamento Da Sua Barbearia}. 2025. Disponível em: \url{https://www.idealsales.com.br/blog/como-o-crm-pode-fidelizar-clientes-e-aumentar-o-faturamento-da-sua-barbearia/}. Acesso em: 12/09/2025.

    \bibitem{ijirt}
    IJIRT. \textit{Design and Implementation of an Online Appointment System for Barber Shops}. [S.l.], [s.d.]. Disponível em: \url{https://ijirt.org/publishedpaper/IJIRT177309_PAPER.pdf}. Acesso em: 07/09/2025.

    \bibitem{ijraset2023}
    IJRASET. \textit{Barber Shop Management System}. 2023. Disponível em: \url{https://www.ijraset.com/best-journal/barber-shop-management-system}. Acesso em: 18/09/2025.

    \bibitem{mutual2025}
    MUTUAL TECNOLOGIA. \textit{Inovações em Agendamentos para Barbearias}. 2025. Disponível em: \url{https://mutualtecnologia.com/inovacoes-em-agendamentos-para-barbearias/}. Acesso em: 12/09/2025.

    \bibitem{oraclejava}
    ORACLE. \textit{Java SE Documentation}. Redwood Shores, CA, [c2024]. Disponível em: \url{https://docs.oracle.com/en/java/}.

    \bibitem{lombok}
    PROJECT LOMBOK. \textit{Features}. [S.l.]: [c2024]. Disponível em: \url{https://projectlombok.org/features/}.

    \bibitem{unifenas2023}
    REVISTA CIENTÍFICA DA UNIFENAS. \textit{APLICATIVO. SALÃO DE BELEZA. BARBEARIA. AGENDAMENTO ONLINE}. 2023. Disponível em: \url{https://revistas.unifenas.br/index.php/revistaunifenas/article/view/875}. Acesso em: 08/10/2025.

    \bibitem{sebrae2023}
    SEBRAE. \textit{Softwares e aplicativos facilitam e melhoram a gestão do salão}. 2023. Disponível em: \url{https://sebrae.com.br/sites/PortalSebrae/artigos/softwares-e-aplicativos-facilitam-e-melhoram-a-gestao-do-salao,f06cac941b896810VgnVCM1000001b00320aRCRD}. Acesso em: 12/09/2025.

    \bibitem{springboot}
    SPRING. \textit{Spring Boot Reference Documentation}. [S.l.]: VMware, [c2024]. Disponível em: \url{https://docs.spring.io/spring-boot/docs/current/reference/htmlsingle/}. Acesso em: 29/08/2025.

    \bibitem{springdata}
    SPRING. \textit{Spring Data JPA Reference Documentation}. [S.l.]: VMware, [c2024]. Disponível em: \url{https://docs.spring.io/spring-data/jpa/docs/current/reference/html/}. Acesso em: 01/09/2025.

    \bibitem{springsecurity}
    SPRING. \textit{Spring Security Documentation}. [S.l.]: VMware, [c2024]. Disponível em: \url{https://docs.spring.io/spring-security/reference/index.html}. Acesso em: 01/09/2025.

    \bibitem{springdoc}
    SPRINGDOC. \textit{Springdoc-OpenAPI}. [S.l.]: [c2024]. Disponível em: \url{https://springdoc.org/}. Acesso em: 02/09/2025.

    \bibitem{trinks2025}
    TRINKS. \textit{Agenda online para barbearia: o que é e como funciona?}. 2025. Disponível em: \url{https://blog.trinks.com/agenda-online-para-barbearia-o-que-e-e-como-funciona/}. Acesso em: 07/09/25.
\end{thebibliography}

% --- APÊNDICES ---
\begin{apendicesenv}

\chapter{Dicionário de Dados}

\section*{Tabela: barbershops}
\textbf{Descrição:} Armazena informações sobre os estabelecimentos (barbearias) cadastrados.
\begin{description}
    \item[id:] VARCHAR(36). Chave Primária (PK). Identificador único universal (UUID) da barbearia.
    \item[name:] VARCHAR(255). Obrigatório (NOT NULL). Nome comercial da barbearia.
    \item[cnpj:] VARCHAR(14). Obrigatório, Único (UNIQUE). Número do CNPJ.
    \item[address:] VARCHAR(255). Endereço completo da barbearia.
    \item[date\_created:] DATETIME. Obrigatório. Data e hora de criação do registro.
    \item[last\_updated:] DATETIME. Obrigatório. Data e hora da última atualização.
    \item[logo\_url:] VARCHAR(255). URL para o arquivo de logo da barbearia.
    \item[banner\_url:] VARCHAR(255). URL para o arquivo de banner da barbearia.
\end{description}

\section*{Tabela: barbers}
\textbf{Descrição:} Armazena informações dos barbeiros (incluindo proprietários).
\begin{description}
    \item[id:] VARCHAR(36). Chave Primária (PK). Identificador único (UUID) do barbeiro.
    \item[name:] VARCHAR(70). Obrigatório. Nome completo do barbeiro.
    \item[tell:] VARCHAR(11). Obrigatório, Único. Número de telefone/celular.
    \item[email:] VARCHAR(70). Obrigatório, Único. Endereço de e-mail (usado para login).
    \item[document\_cpf:] VARCHAR(11). Obrigatório, Único. Número do CPF.
    \item[password:] VARCHAR(255). Obrigatório. Senha criptografada (hash).
    \item[barbershop\_id:] VARCHAR(36). Chave Estrangeira (FK) de barbershops. Indica a barbearia vinculada. Permite NULL.
    \item[is\_owner:] TINYINT(1). Obrigatório. Flag booleana (1 para proprietário, 0 para funcionário).
    \item[date\_created:] DATETIME. Obrigatório. Data e hora de criação.
    \item[last\_updated:] DATETIME. Obrigatório. Data e hora da última atualização.
    \item[work\_start\_time:] TIME. Hora de início da jornada de trabalho.
    \item[work\_end\_time:] TIME. Hora de término da jornada de trabalho.
    \item[image\_url:] VARCHAR(255). URL para a foto de perfil do barbeiro.
\end{description}

\section*{Tabela: customers}
\textbf{Descrição:} Armazena informações dos clientes usuários da plataforma.
\begin{description}
    \item[id:] VARCHAR(36). Chave Primária (PK). Identificador único (UUID) do cliente.
    \item[name:] VARCHAR(70). Obrigatório. Nome completo do cliente.
    \item[tell:] VARCHAR(11). Obrigatório, Único. Número de telefone/celular.
    \item[email:] VARCHAR(70). Obrigatório, Único. Endereço de e-mail (usado para login).
    \item[document\_cpf:] VARCHAR(11). Obrigatório, Único. Número do CPF.
    \item[password:] VARCHAR(255). Obrigatório. Senha criptografada (hash).
    \item[date\_created:] DATETIME. Obrigatório. Data e hora de criação.
    \item[last\_updated:] DATETIME. Obrigatório. Data e hora da última atualização.
    \item[image\_url:] VARCHAR(255). URL para a foto de perfil do cliente.
\end{description}

\section*{Tabela: activities}
\textbf{Descrição:} Armazena a lista de serviços oferecidos por uma barbearia.
\begin{description}
    \item[id:] VARCHAR(36). Chave Primária (PK). Identificador único (UUID) da atividade/serviço.
    \item[barbershop\_id:] VARCHAR(36). Chave Estrangeira (FK) de barbershops. Obrigatório.
    \item[activity\_name:] VARCHAR(255). Obrigatório. Nome do serviço (Ex: "Corte de Cabelo").
    \item[price:] DECIMAL(10,2). Obrigatório. Preço do serviço.
    \item[duration\_minutes:] INT. Obrigatório. Duração estimada do serviço em minutos.
    \item[date\_created:] DATETIME. Obrigatório. Data e hora de criação.
    \item[last\_updated:] DATETIME. Obrigatório. Data e hora da última atualização.
    \item[image\_url:] VARCHAR(255). URL para uma imagem representativa do serviço.
\end{description}

\section*{Tabela: appointments}
\textbf{Descrição:} Armazena o registro de cada agendamento.
\begin{description}
    \item[id:] BIGINT. Chave Primária (PK), Auto Incremento (AI).
    \item[barbershop\_id:] VARCHAR(36). Chave Estrangeira (FK) de barbershops. Obrigatório.
    \item[barber\_id:] VARCHAR(36). Chave Estrangeira (FK) de barbers. Obrigatório.
    \item[customer\_id:] VARCHAR(36). Chave Estrangeira (FK) de customers. Obrigatório.
    \item[start\_time:] DATETIME. Obrigatório. Data e hora de início do agendamento.
    \item[end\_time:] DATETIME. Obrigatório. Data e hora de término do agendamento.
    \item[status:] VARCHAR(50). Obrigatório (Padrão: SCHEDULED).
    \item[date\_created:] DATETIME. Obrigatório. Data e hora de criação.
    \item[last\_updated:] DATETIME. Obrigatório. Data e hora da última atualização.
\end{description}

\section*{Tabela: appointment\_activities}
\textbf{Descrição:} Tabela de ligação N:M (agendamentos $\leftrightarrow$ activities).
\begin{description}
    \item[id:] BIGINT. Chave Primária (PK), Auto Incremento (AI).
    \item[appointment\_id:] BIGINT. Chave Estrangeira (FK) de appointments. Obrigatório.
    \item[activity\_id:] VARCHAR(36). Chave Estrangeira (FK) de activities.
\end{description}

\section*{Tabela: barber\_activities}
\textbf{Descrição:} Tabela de ligação N:M (barbers $\leftrightarrow$ activities).
\begin{description}
    \item[barber\_id:] VARCHAR(36). Chave Primária Composta, FK de barbers. Obrigatório.
    \item[activity\_id:] VARCHAR(36). Chave Primária Composta, FK de activities. Obrigatório.
\end{description}

\section*{Tabela: barbershop\_join\_requests}
\textbf{Descrição:} Armazena os pedidos de barbeiros para se vincular a uma barbearia.
\begin{description}
    \item[id:] BIGINT. Chave Primária (PK), Auto Incremento (AI).
    \item[barber\_id:] VARCHAR(36). Chave Estrangeira (FK) de barbers. Obrigatório.
    \item[barbershop\_id:] VARCHAR(36). Chave Estrangeira (FK) de barbershops. Obrigatório.
    \item[status:] ENUM('PENDING','REJECTED'). Obrigatório (Padrão: PENDING).
    \item[date\_created:] DATETIME. Obrigatório. Data e hora da solicitação.
\end{description}

\section*{Tabela: barbershop\_highlights}
\textbf{Descrição:} Armazena imagens de destaque ou portfólio de uma barbearia.
\begin{description}
    \item[id:] VARCHAR(36). Chave Primária (PK). Identificador único (UUID) do destaque.
    \item[image\_url:] VARCHAR(255). Obrigatório. URL para a imagem de destaque.
    \item[barbershop\_id:] VARCHAR(36). Chave Estrangeira (FK) de barbershops. Obrigatório.
\end{description}

\end{apendicesenv}

% --- FIM DO DOCUMENTO ---
\end{document}
